\documentclass{article}

\usepackage{sectsty}      % for section heading

\usepackage[ttscale=.875]{libertine} % font
\usepackage[T1]{fontenc}  % for 8-bit encoding

\usepackage{enumitem}     % for customizing itemize
\usepackage{fontawesome5} % for contact info logos
\usepackage{hyperref}     % for link
\usepackage{mdwlist}      % for compact lists (itemize*)

\usepackage[letterpaper,margin=1in]{geometry}


\sectionfont{\normalfont\sectionrule{0pt}{0pt}{-4pt}{1pt}}

% format two pieces of text: one left aligned and one right aligned
\newcommand{\headerrow}[2]
{\begin{tabular*}{\linewidth}{l@{\extracolsep{\fill}}r}
    #1 &
    #2 \\
\end{tabular*}}

\newcommand{\email}[1]{
    \faEnvelope \space #1
}

\newcommand{\phone}[1]{
    \faPhone \space #1
}

\newcommand{\github}[1]{
    \href{https://www.github.com/#1}{\faGithub \space #1}
}

\newcommand{\linkedin}[1]{
    \href{https://www.linkedin.com/in/#1}{\faLinkedin \space #1}
}

\begin{document}
\begin{center}
    {\huge \textsc{\textbf{Rahul Venkatesh}}} \\[2pt]
    % Paris $\mid$ \phone {+33 7 xx xx xx xx} $\mid$ \email{rahulv.9942@gmail.com}
    \phone{+65 xxxx xxxx}
    $\mid$
    \email{rahulv.9942@gmail.com}
    $\mid$
    \linkedin{recurze}
    $\mid$
    \github{recurze}
\end{center}

\section*{\textsc{Education}}
\begin{itemize}[leftmargin=0em]

\item[]
    \headerrow {\textbf{National University of Singapore}}{Singapore}
    \headerrow {\emph{Master of Science in Data Science and Machine Learning}}{\emph{2023-Present}}

\item[]
    \headerrow {\textbf{Indian Institute of Technology (IIT), Delhi}}{New Delhi, India}
    \headerrow {\emph{Bachelor of Technology in Computer Science and Engineering}}{\emph{2016-2020}}

\end{itemize}

%\section*{Scholastic Achievements}
%\begin{itemize}
%    \item
%        Selected for \textbf{ACM-ICPC $2018$} Regional Round with teams
%        competing from all over India.
%    \item
%        Secured \textbf{All India Rank - $2754$} in Joint Entrance Exam
%        (Advanced) - $2016$, among $150$k candidates.
%    \item
%        Awarded Kishore Vaigyanik Protsahan Yojana \textbf{(KVYP) Fellowship} by
%        Indian Institute of Science, $2015$.
%    \item
%        Received National Talent Search \textbf{(NTS) Scholarship} for being
%        in the top 1000 at National Level, $2012$.
%\end{itemize}
%

\section*{\textsc{Technical Skills}}
\begin{itemize}[itemsep=0em, leftmargin=0.7em]
\item[]
    \textbf{Languages}: C++, Python, SQL, Bash
\item[]
    \textbf{Tools/Software}: TensorFlow, OpenMP, CUDA, Git, GDB
\end{itemize}

\section*{\textsc{Work Experience}}
\begin{itemize}[leftmargin=0em]
\parskip=0.1em

\item[]
    \headerrow {\textbf{Squarepoint Capital}}{Paris, France}
    \headerrow {\emph{Software Engineering}}{\emph{Aug 2020 - Jun 2023}}

    \begin{itemize*}
        \item
            Developed and supported (in production) \textbf{low-latency} order entry gateways (OEG) for algorithmic trading.
        \item
            Built new OEGs to $5+$ \textbf{exchanges} including \textbf{CME, ICE} and \textbf{OSE}
            over various protocols like OUCH and FIX.
        \item
            Implemented frameworks for trading $2$ new asset classes:
            \textbf{Bonds} and \textbf{Non-deliverable forwards (NDF)}.
        \item
            Coordinated with QA and delivered numerous new projects (gateways) and business requests (features).
        \item
            \textbf{Improved performance} of the gateway application by
            \textbf{collecting and analyzing} latency-related \textbf{data}.
    \end{itemize*}
\end{itemize}

\section*{\textsc{Projects}}
\begin{itemize}[leftmargin=0em]

\item[]
    \headerrow {\textbf{Parallel Laplacian Solver}}{Prof. Amitabha Bagchi}
    \headerrow {\emph{Algorithms Project}}{\emph{Aug 2019 - Dec 2019}}

    \begin{itemize*}
        \item
            Implemented a novel \textbf{random-walk distributed} method to solve an
            important class of Laplacian Systems.
        \item
            Leveraged C++/OpenMP API for parallelization, and \texttt{gprof} and \texttt{valgrind} for \textbf{profiling}.
        \item
            Improved performance on sparse graphs with \textbf{optimizations} like Alias Method and "densifying".
        \item
            Visualized performance against existing solvers like Dan Spielman's and Bechetti's: \\
            Overall, it's \textbf{better than} Bechetti's. With lot of parameter fine-tuning, it's \textbf{comparable with} Dan's.
    \end{itemize*}

%\item[]
%    \headerrow {\textbf{Public Credit Registry}}{Prof. Subhashis Banerjee, Prof.
%        Subodh Sharma}
%    \headerrow {\emph{Summer Undergraduate Project}}{\emph{May 2019 - Jul 2019}}
%
%    \begin{itemize*}
%        \item
%            Constructed first-order logical representation of data
%            privacy, security and other correctness properties.
%        \item
%            Proposed a complete example of a design with unit tests and
%            end-to-end testing of the model.
%        \item
%            Employed blockchain technology similar to \textbf{Hyperledger
%            Fabrics} and \textbf{Software Guard Extension}.
%    \end{itemize*}

\item[]
    \headerrow {\textbf{AI Game-playing bot}}{Prof. Mausam}
    \headerrow {\emph{Artificial Intelligence Project}}{\emph{Oct 2018}}

    \begin{itemize*}
        \item
            Created a bot to play Yinsh, a $2$-player game on an hexagonal board
            with a branching factor of $30$.
        \item
            Implemented \textbf{Alpha-beta pruning} with \textbf{static move ordering}
            to choose next move from all possible moves.
        \item
            Optimized move generation using \textbf{Bitboards} and \textbf{Zobrist Hashing},
            boosting the search to $4$ moves ahead.
        \item
            Fine-tuned weights for explicit hand-crafted features through automated trial-and-error.
    \end{itemize*}

%\item
%    Toy prolog interpreter
%    $\mid$
%    Prof. Sanjiva Prasad
%    \hfill
%    MAR $2018$
%
%    \begin{itemize}
%        \item
%            Built a interpreter for logic programming language - Prolog, capable
%            of solving prolog predicates.
%        \item
%            The front end of the interpreter consitutes of the \textbf{Lex
%            scanner} (ocamllex) and \textbf{LALR parser} (ocamlyacc).
%        \item
%            The scanner converts the program into tokens which in turn are
%            converted into \textbf{Abstract Syntax Tree}.
%        \item
%            A explicit \textbf{backtracking algorithm} using fixed size stacks
%            was implemented to solve the predicates.
%    \end{itemize}

\end{itemize}

\section*{\textsc{Relevant Coursework}}
\begin{itemize}[itemsep=0em, leftmargin=0.7em]
\item[]
    \textbf{Mathematics}: Linear Algebra, Calculus, Probability and Stochastic Processes
\item[]
    \textbf{Computer Science}: Artificial Intelligence, Machine Learning, Advanced Algorithms
\end{itemize}

%\section*{Positions of Responsibility}
%\begin{itemize}
%    \item
%        As Vice-Captain, led the hostel chess team to be placed first in
%        Inter-Hostel Chess Tournament 2017 at IIT.
%    \item
%        As Captain of the Red House, ensured we stood first in the annual
%        intra-high school extracurricular competitions.
%\end{itemize}

%\section*{\textsc{Interests, Talents and Skills}}
%\begin{itemize*}
%    \item
%        Chess - FIDE Rating: $1380$
%    \item
%        Basketball - Represented the school team and hostel team at IIT Delhi.
%    \item
%        Competitive programming (uname: recurze) – Ratings: Codeforces - $1813$
%        and Codechef - $1877$.
%\end{itemize*}

\end{document}
