\documentclass{article}

\usepackage{sectsty}
\usepackage{enumitem}
\usepackage{tabularx}
\usepackage[letterpaper,margin=0.70in]{geometry}

\sectionfont{\normalfont\sectionrule{0pt}{0pt}{-4pt}{1pt}}

\begin{document}
\begin{center} {\Large \textsc{Rahul Venkatesh}}\end{center}

\section*{Academic Details}
\begin{center} \begin{tabular}{ | c | c | c | c |}
    \hline

    \textbf{Year} &
    \textbf{Degree} &
    \textbf{Institute} &
    \textbf{CGPA/Percentage} \\

    \hline

    2016-2020 &
    B.Tech in Computer Science &
    Indian Institute of Technology, Delhi &
    7.370 \\

    \hline

    2016 &
    Class XII,
    Karnataka State &
    Mother Teresa PU College, Mysuru &
    92.1\% \\

    \hline
\end{tabular} \end{center}

\section*{Scholastic Achievements}
\begin{itemize}[noitemsep,nolistsep]
    \item
        Selected for \textbf{ACM-ICPC $2018$} Regional Round with teams
        competing from all over India.
    \item
        Secured \textbf{All India Rank - $2754$} in Joint Entrance Exam
        (Advanced) - $2016$, among $150,000$ candidates.
    \item
        Awarded Kishore Vaigyanik Protsahan Yojana \textbf{(KVYP) Fellowship} by
        Indian Institute of Science, $2015$.
    \item
        Received National Talent Search \textbf{(NTS) Scholarship} for being
        in the top 1000 at National Level, 2012.
\end{itemize}

\section*{Work Experience}
\begin{enumerate}
\item
    Order Entry Gateway
    $\mid$
    Squarepoint Capital
    \hfill
    AUG 2020 - PRESENT

    \begin{itemize}[noitemsep,nolistsep]
        \item
            Developed, tested, debugged and documented applications used for
            algorithmic trading.
        \item
            Built new order entry gateways to various stock exchanges over different order entry protocols.
        \item
            Improved \textbf{performance} of the application by
            \textbf{collecting and analyzing} latency-related \textbf{data}.
    \end{itemize}

\end{enumerate}

\section*{Projects}
\begin{enumerate}
\item
    Parallel Laplacian Solver
    $\mid$
    Prof. Amitabha Bagchi
    \hfill
    AUG $2019$

    \begin{itemize}[noitemsep,nolistsep]
        \item
            Implemented a \textbf{random-walk distributed} method to solve an
            important class of Laplacian Systems.
        \item
            Experimented with various \textbf{optimizations} and used
            \textbf{Alias Method} to successfully improve performance.
        \item
            Wrote and tested in C++; used OpenMP application programming
            interface for parallelization.
    \end{itemize}

\item
    Public Credit Registry
    $\mid$
    Prof. Subhashis Banerjee, Prof. Subodh Sharma
    \hfill
    MAY $2019$

    \begin{itemize}[noitemsep,nolistsep]
        \item
            Constructed first-order logical representation of data
            privacy, security and other correctness properties.
        \item
            Proposed a complete example of a design with unit tests and
            end-to-end testing of the model.
        \item
            Employed blockchain technology similar to \textbf{Hyperledger
            Fabrics} and \textbf{Software Guard Extension}.
    \end{itemize}

\item
    AI Game-playing bot
    $\mid$
    Prof. Mausam
    \hfill
    OCT $2018$

    \begin{itemize}[noitemsep,nolistsep]
        \item
            Created a bot to play Yinsh, a $2-$player game on an hexagonal board
            with a branching factor of $30$.
        \item
            Used \textbf{Alpha beta pruning} with \textbf{static move ordering}
            to decide next move from set of all possible moves.
        \item
            Utilized \textbf{Bitboards} and \textbf{Zobrist Hashing} to speed up
            move generation, thus searching $4$ moves ahead.
    \end{itemize}

%\item
%    Toy prolog interpreter
%    $\mid$
%    Prof. Sanjiva Prasad
%    \hfill
%    MAR $2018$
%
%    \begin{itemize}[noitemsep,nolistsep]
%        \item
%            Built a interpreter for logic programming language - Prolog, capable
%            of solving prolog predicates.
%        \item
%            The front end of the interpreter consitutes of the \textbf{Lex
%            scanner} (ocamllex) and \textbf{LALR parser} (ocamlyacc).
%        \item
%            The scanner converts the program into tokens which in turn are
%            converted into \textbf{Abstract Syntax Tree}.
%        \item
%            A explicit \textbf{backtracking algorithm} using fixed size stacks
%            was implemented to solve the predicates.
%    \end{itemize}

\end{enumerate}

\section*{Relevant Coursework}
\begin{itemize}[noitemsep,nolistsep]
\item
    Math: Linear Algebra and Differential Equations, Calculus, Probability and
        Stochastic Processes
\item
    Computer Science: Data Structures and Algorithms, Principles of Artificial Intelligence, Machine Learning
\end{itemize}

\section*{Technical Skills}
\begin{itemize}[noitemsep,nolistsep]
\item
    Programming Languages: C/C$++$, Python, Bash
\item
    Tools/Software: PyTorch, TensorFlow, OpenMP, CUDA
\end{itemize}

%\section*{Positions of Responsibility}
%\begin{itemize}[noitemsep,nolistsep]
%    \item
%        As Vice-Captain, led the hostel chess team to be placed first in
%        Inter-Hostel Chess Tournament 2017 at IIT.
%    \item
%        As Captain of the Red House, ensured we stood first in the annual
%        intra-high school extracurricular competitions.
%\end{itemize}

%\section*{Interests, Talents and Skills}
%\begin{itemize}[noitemsep,nolistsep]
%    \item
%        Chess - FIDE Rating: $1380$
%    \item
%        Basketball - Represented the school team and currently hostel team at
%        IIT Delhi.
%    \item
%        Competitive programming (uname: recurze) – Ratings: Codeforces - $1813$
%        and Codechef - $1877$.
%\end{itemize}

\end{document}
